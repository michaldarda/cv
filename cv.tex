%%%%%%%%%%%%%%%%%
% This is an sample CV template created using altacv.cls
% (v1.1.5, 1 December 2018) written by LianTze Lim (liantze@gmail.com). Now compiles with pdfLaTeX, XeLaTeX and LuaLaTeX.
%
%% It may be distributed and/or modified under the
%% conditions of the LaTeX Project Public License, either version 1.3
%% of this license or (at your option) any later version.
%% The latest version of this license is in
%%    http://www.latex-project.org/lppl.txt
%% and version 1.3 or later is part of all distributions of LaTeX
%% version 2003/12/01 or later.
%%%%%%%%%%%%%%%%

%% If you need to pass whatever options to xcolor
\PassOptionsToPackage{dvipsnames}{xcolor}

%% If you are using \orcid or academicons
%% icons, make sure you have the academicons
%% option here, and compile with XeLaTeX
%% or LuaLaTeX.
% \documentclass[10pt,a4paper,academicons]{altacv}

%% Use the "normalphoto" option if you want a normal photo instead of cropped to a circle
% \documentclass[10pt,a4paper,normalphoto]{altacv}

\documentclass[10pt,a4paper,ragged2e]{altacv}

%% AltaCV uses the fontawesome and academicon fonts
%% and packages.
%% See texdoc.net/pkg/fontawecome and http://texdoc.net/pkg/academicons for full list of symbols. You MUST compile with XeLaTeX or LuaLaTeX if you want to use academicons.

% Change the page layout if you need to
\geometry{left=1cm,right=9cm,marginparwidth=6.8cm,marginparsep=1.2cm,top=1.25cm,bottom=1.25cm}

% Change the font if you want to, depending on whether
% you're using pdflatex or xelatex/lualatex
\ifxetexorluatex
  % If using xelatex or lualatex:
  \setmainfont{Lato}
\else
  % If using pdflatex:
  \usepackage[utf8]{inputenc}
  \usepackage[T1]{fontenc}
  \usepackage[default]{lato}
\fi

% Change the colours if you want to
\definecolor{Mulberry}{HTML}{72243D}
\definecolor{SlateGrey}{HTML}{2E2E2E}
\definecolor{LightGrey}{HTML}{666666}
\colorlet{heading}{Sepia}
\colorlet{accent}{Mulberry}
\colorlet{emphasis}{SlateGrey}
\colorlet{body}{LightGrey}

% Change the bullets for itemize and rating marker
% for \cvskill if you want to
\renewcommand{\itemmarker}{{\small\textbullet}}
\renewcommand{\ratingmarker}{\faCircle}

%% sample.bib contains your publications
\addbibresource{sample.bib}

\begin{document}
\name{Michał Darda}
\tagline{Software Developer}
% \photo{2.8cm}{Globe_High}
\personalinfo{%
  % Not all of these are required!
  % You can add your own with \printinfo{symbol}{detail}
  \email{mdarda@pm.me}
  \phone{514-455-097}
  % \mailaddress{Address, Street, 00000 County}
  \location{Warsaw, Poland}
  % \homepage{www.homepage.com/}
  % \twitter{@twitterhandle}
  \linkedin{linkedin.com/in/mdarda}
  \github{github.com/michaldarda}
  %% You MUST add the academicons option to \documentclass, then compile with LuaLaTeX or XeLaTeX, if you want to use \orcid or other academicons commands.
  % \orcid{orcid.org/0000-0000-0000-0000}
}

%% Make the header extend all the way to the right, if you want.
\begin{fullwidth}
\makecvheader
\end{fullwidth}

%% Depending on your tastes, you may want to make fonts of itemize environments slightly smaller
% \AtBeginEnvironment{itemize}{\small}

%% Provide the file name containing the sidebar contents as an optional parameter to \cvsection.
%% You can always just use \marginpar{...} if you do
%% not need to align the top of the contents to any
%% \cvsection title in the "main" bar.
\cvsection[sidebar]{Experience}

\cvevent{Software Engineer 4 (out of 6 levels)}{Appriss Health}{January 2019 -- Ongoing}{Warsaw, Poland}
\begin{itemize}
\item Maintaining set of applications intended to deliver reports for pharmasists and physicians \newline
Project for U.S. public healthcare sector \newline
{\small Stack: Ruby, Rails, Vue.js, PostgreSQL, Redis, Sidekiq Enterprise}
\end{itemize}

\divider

\cvevent{Full Stack Developer}{Educartis}{October 2017 -- December 2018}{Warsaw, Poland}
\begin{itemize}
\item Developing educartis.com multi-country platform \newline
{\small Stack: Ruby, Rails, Vue.js, PostgreSQL, Heroku, Sidekiq}
\end{itemize}

\divider

\cvevent{Full Stack Developer}{Kreditech}{May 2016 -- October 2017}{Warsaw, Poland}
\begin{itemize}
\item Developing Monedo Pay payment application
\item Developing internal ad management system
\item Developing microservice for handling customers communication \newline
{\small Stack: Ruby, Rails, Go, React.js, PostgreSQL, MongoDB, RabbitMQ}
\end{itemize}

\divider

\cvevent{Ruby Developer}{Cubiware}{August 2014 -- April 2016}{Remote}
\begin{itemize}
\item Working on services in Rails, plain Ruby offering Admin Panel,
XML RPC API, JSON API, XML REST API \newline
{\small Stack: Ruby, Sinatra, Rails, MySQL, RabbitMQ}
\end{itemize}

\divider

\cvevent{Ruby On Rails Developer}{Positionly}{August 2014 -- August 2014}{Warsaw, Poland}
\begin{itemize}
\item Developing positionly.com core application \newline
{\small Stack: Ruby, Rails, PostgreSQL, MongoDB, Resque}
\end{itemize}

\cvsection{Projects}

\cvevent{Landbook}{}{}{May 2016 - August 2016}
\begin{itemize}
\item Developed land-book.com using Ruby On Rails, jQuery.
\end{itemize}

\divider

\cvevent{Polak programming language}{}{}{2013}
\begin{itemize}
\item Open source programming language written in Ruby as
school project. Entire syntax of language is in polish. \newline
github.com/michaldarda/polak
\end{itemize}

\medskip

\clearpage

\nocite{*}

%% If the NEXT page doesn't start with a \cvsection but you'd
%% still like to add a sidebar, then use this command on THIS
%% page to add it. The optional argument lets you pull up the
%% sidebar a bit so that it looks aligned with the top of the
%% main column.
% \addnextpagesidebar[-1ex]{page3sidebar}


\end{document}
